\title{Assignment 1 -- CSC 225}
\author{
	Andrew \bf{Hobden} \\
		Student Number: \bf{V00788452}\\
	Instructor: Venkatesh Srinivasan
}
\date{\today}

\documentclass[12pt]{article}
\usepackage{mathtools}
\usepackage{amssymb}

\begin{document}
\maketitle

% Excercise 1.4.8
\section{Excercise 1.4.8}
\subsection{Psuedocode}
\begin{verbatim} 
	function(input) =
	  result ← 0
	  # Sort the array
	  sortedInput ← sort(input)
	  last <- 0
	  count <- 1 
	  for item in sortedInput
	    if item is last
	      count++
	    else
	      result <- result + (count*(count-1))/2
	      count <- 1
	      last <- item
	  end
	  result <- result + (count*(count-1))/2 # Cover the last item!
	  return result
\end{verbatim}

\clearpage
\subsection{CoffeeScript Example}
\begin{verbatim}
	question = (input) ->
	  result = 0
	  sortedInput = input.sort() # O(n log n)
	  last = 0
	  count = 1
	  for item in sortedInput
	    if item is last
	      count++
	    else
	      result += (count*(count-1))/2
	      count = 1
	      last = item
	  result += (count*(count-1))/2 # Cover our last item too!
	  return result
\end{verbatim}

\subsection{Explanation}
The complexity of this function should be $ O(n \log n ) $ since:
\paragraph{Merge Sort} $ O(n \log n)$
\paragraph{Single Loop} $ O(n) $

% Excercise 1.4.12
\section{Excercise 1.4.12}
\subsection{Psuedocode}
\begin{verbatim}
	function(input1, input2) =
	  # Initialize an empty result
	  result <- []
	  # Since our arrays are already sorted, we can just iterate
	  for i <- 0 to (input1.length -1)
	    if input[i] = input[i+1]
	      result.push( input1[i] )
	  end
	return result
	
\end{verbatim}

\subsection{Coffeescript Example}
\begin{verbatim}
	question = (input1, input2) ->
	  # Initialize an empty result
	  result = []
	  # Since our arrays are already sorted, we can just iterate
	  for i in [0 .. (input1.length - 1)]
	    if input1[i] is input2[i]
	      result.push(input1[i])
	  return result
\end{verbatim}

\subsection{Explanation}
The complexity of this function should be $ O(n) $ since:
\paragraph{Single Loop} $ O(n) $

\section{Excercise 1.4.6}
\subsection{Problem 1}
\begin{verbatim}
	int sum = 0;
	for (int n = N; n > 0; n /= 2)
	  for (int i = 0; i < n; i++)
	    sum++;
\end{verbatim}
This code is $ O(n \log n) $, since the first loop will run $ \log n $ times, and the second loop will occur, in worst case, $ n $ times.

\subsection{Problem 2}
\begin{verbatim}
	int sum = 0;
	for (int i = 1; i < N; i *= 2)
	  for (int j = 0; i < i; j++)
	    sum++;
\end{verbatim}
This code is $ O(n \log n) $, since the first loop runs $ \log n $ times, and the second loop occurs, in worst case, $ n $ times.

\subsection{Problem 3}
\begin{verbatim}
	int sum = 0;
	for (int i = 1 i < N; i *= 2)
	  for (int j = 0; j < N; j++)
	    sum++;
\end{verbatim}
This code is $ O(n \log n) $, since the first loop runs $ \log n $ times, and the second loop runs $ n $ times.

\section{Ordering Functions by Runtime}
Ordering from shortest to longest runtimes, since it was not specified.
\begin{enumerate}
	\item $ 3^{100} \equiv O(1) $
	\item $ \log \log n \equiv O(\log \log n) $
	\item $ (\log n)^2 \equiv O(\log^2 n) $
	\item $ 2n \equiv O(n) $
	\item $ 5n \equiv O(n) $
	\item $ 4^{\log n} \equiv O(4^{\log n}) $
	\item $ n! \equiv O(n!) $
\end{enumerate}

\section{Excercise 1.4.1}
Let $ f(n) = $ The number of triples that can be chosen from $n$ items. \\
\subsection{Showing that}
\begin{displaymath}
	f(n) = \frac{n(n-1)(n-2)}{6}
\end{displaymath}
\subsection{Base Case:}
We expect to get a result of 1 if we allow $ n = 3 $.
\begin{displaymath}
	\frac{3(3-1)(3-2)}{6} = \frac{6}{6} = 1
\end{displaymath}
$ \therefore $ True for $ n = 0 $
\subsection{Inductive Hypothesis}
Suppose that $ f(n) = \frac{n(n-1)(n-2)}{6} $ for some $ n \geq 3 $
\subsection{Inductive Step}
Consider $ n+1 $.
\begin{displaymath}
	f(n+1)=f(n)*\frac{n+1}{n-2}
\end{displaymath}
Then, by the inductive hypothesis
\begin{displaymath}
	f(n)*\frac{n+1}{n-2}=\frac{n(n-1)(n-2)}{6}*\frac{n+1}{n-2}
\end{displaymath}
Simplifying to
\begin{displaymath}
	\frac{n(n-1)(n-2)}{6}*\frac{n+1}{n-2} = \frac{(n+1)(n)(n-1)(n-2)}{(n-2)6}
\end{displaymath}
Cancelling $ (n-2) $ we find our statement
\begin{displaymath}
	\frac{(n+1)(n)(n-1)}{6} = \frac{(n+1)((n+1)-1)((n+1)-2)}{6}
\end{displaymath}
$ \therefore $ by induction, the statement holds $ \forall n \geq 0 $

\section{Mathematical Induction Proof}
\subsection{Showing that}
\begin{displaymath}
	1+2+3+...+n=\frac{n(n+1)}{2}
\end{displaymath}
\subsection{Base Case}
We expect to get a result of $ 0 $ for $ n = 0 $.
\begin{displaymath}
	\frac{0(0+1)}{2}=0
\end{displaymath}
$ \therefore $ True for $ n = 0 $
\subsection{Inductive Hypothesis}
Suppose that $ \sum_{i=0}^{n} i = \frac{n(n+1)}{2} $ for some $ n \geq 0 $.
\subsection{Inductive Step}
Consider $ n+1 $.
\begin{displaymath}
	\sum_{i=0}^{n+1} i = \sum_{i=0}^{n} i + (n+1)
\end{displaymath}
Then, by the inductive hypothesis
\begin{displaymath}
	\sum_{i=0}^{n} i+(n+1) = \frac{n(n+1)}{2} + (n+1)
\end{displaymath}
Which simplifies down to our statement
\begin{displaymath}
	\frac{n(n+1)+2(n+1)}{2}=\frac{n^2+3n+2}{2}=\frac{(n+1)((n+1)+1)}{2}
\end{displaymath}
$ \therefore $ by induction, the statement holds $ \forall n \geq 0 $

\end{document}